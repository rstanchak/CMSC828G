\documentclass[11pt]{article}
\usepackage{homework}
%
\begin{document}
%
\title{Questions for Airoldi}
\author{Roman Stanchak}
\maketitle

\begin{enumerate}
	\item It seems that a result of the submodularity property is that the
		cascades being considered are, in a sense, trees. The fact that links
		have directionality allows the graph to be partitioned into sections
		that don't influence one another. Does this approach work with graphs
		with cycles or dense graphs with a single connected component?
	\item Similarly, does the approach apply to SIS infection models (where nodes can become susceptible again after recovery)?
	\item The authors report their results on the Battle of Water Sensor Networks (BWSN) data ¿ how did their method fare in the "battle"?
\end{enumerate}
\end{document}
